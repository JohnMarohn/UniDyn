\documentclass[12pt,letterpaper]{refart}

\usepackage{fancyvrb}                        % verbatim function
\usepackage[usenames,dvipsnames]{color}      % Periwinkle
\usepackage{listings}                        % for typesetting computer code
\usepackage{fancyhdr}                        % fancy page top and bottom
\usepackage{palatino}                        % cool font 
\usepackage[sort&compress,
	square,
	numbers]{natbib}                         % bibliography                        
\usepackage[colorlinks=true,
	citecolor=blue,
	linkcolor=blue,
	urlcolor=blue,
	pagebackref=false]{hyperref}             % hyperrefs
	
\newcommand{\VerbFcn}[1]{{#1}}
\newcommand{\VerbCmd}[1]{{\bf #1}}

\setlength\papermarginwidth{1.50in}
\settextfraction{1.00}

\begin{document}
\sloppy

\title{The UniDyn \emph{Mathematica} package}
\author{
	John A. Marohn\thanks{jam99@cornell.edu} \\ 
	{\footnotesize Dept. of Chemistry and Chemical Biology} \\ 
	{\footnotesize Cornell University, Ithaca, NY 14851-1301, USA}
}
\maketitle

\begin{abstract}
We have developed a \emph{Mathematica} algorithm for symbolically calculating the unitary transformations of quantum-mechanical operators.  The algorithm obtains closed-form analytical results, does not rely on a matrix representation of the operators, and is applicable to both bounded systems like coupled spins and unbounded systems like harmonic oscillators.  Two example calculations are  
\[
	I_{x}(t) 
		= e^{-i \, \omega t \, I_z} \, I_x \, e^{+i \, \omega t \, I_z} 
		= I_x \cos{\omega t} + I_y \sin{\omega t}
\]
and
\[
	a^{\dagger}(t)
		= e^{-i \, \omega t \, a^{\dagger} \, a} a^{\dagger} e^{+i \, \omega t \, a^{\dagger}} 
		= a^{\dagger} \, e^{-i \, \omega t}
\] 
with $(I_x, I_y, I_z)$ the spin angular-momentum operators, $(a^{\dagger}, a)$ the harmonic-oscillator raising and lowering operators, $\omega$ a frequency, and $t$ time.  The rotations are ``self derived'' from the underlying commutation relations, $[I_x, I_y] = i \, I_z$ \& c.p and $[a, a^{\dagger}] = 1$ in these examples.  We call the package \verb+UniDyn+, a mnemonic for \emph{unitary dynamics}.  Example calculations are presented involving magnetic resonance and quantum optics.  
\end{abstract}

\pagestyle{fancy}  % page header and footers

	\lhead{\textsf{UniDyn \emph{Mathematica} package}}
	\rhead{\textsf{John A.\ Marohn}}
	\cfoot{\thepage}
	
\renewcommand{\headrulewidth}{0.4pt} % line at top of page; 0.4pt typical
\renewcommand{\footrulewidth}{0.4pt} % line at bottom of page; 0.4pt typical


\lstset{ %
% language=Mathematica,      % the language of the code -- buggy  
keywords={
	Clear,
	True,False,
	Exp,Sin,Cos,Tan,
	Not,
	Flatten,
	StringJoin, ToString, Length,
	Module, Do, For,
	Map, Table, Position, Sort, Sequence, List,
	Return, 
	Dimensions
	},
keywordstyle={\sffamily\bfseries\footnotesize}, % fonts used for keywords
basicstyle={\sffamily\footnotesize},          % fonts that are used for everything else
xleftmargin=2em,
xrightmargin=-2em,
tabsize=2,                 % sets default tabsize to 2 spaces
captionpos=t,              % sets the caption-position to bottom
caption=\lstname,          % show the filename 
comment=[is]{(**}{**)},    % completely IGNORE text between "(**" and "**)" in the code 
escapeinside={(*@}{@*)},   % code between "(*@" and "@*)" will be converted to LaTeX
%
escapebegin=\color{Periwinkle}\hspace{-2em}\begin{minipage}{\linewidth},
escapeend=\end{minipage}, 
breakindent=-2em,
postbreak=\space,
showstringspaces=false,
breakatwhitespace=true,
%
framexleftmargin=2em,
framexrightmargin=-2em,
frame=lines,
frame=tb,
%
rangeprefix=(*~\ ,  % we will include only code between the lines "(*~ START ~*)" 
rangesuffix=\ ~*),  % ... and "(*~ END ~*)"; the "START" and "END" parts are specified below
columns=flexible,   % this is CRUCIAL; now function names just look like times new roman font....
%
literate={x\$var}{{\emph{x}}}1 
{i\$max\$sym}{{\ensuremath{i_{\mathrm{max}}}}}1
{j\$max\$sym}{{\ensuremath{j_{\mathrm{max}}}}}1
{n\$sym}{{\emph{n}}}1
{i\$sym}{{\emph{i}}}1
{j\$sym}{{\emph{j}}}1
{a\$sym}{{\emph{a}}}1
{b\$sym}{{\emph{b}}}1
{c\$sym}{{\emph{c}}}1
{d\$sym}{{\emph{d}}}1
{Ix\$sym}{{\ensuremath{I_x}}}1
{Iy\$sym}{{\ensuremath{I_y}}}1
{Iz\$sym}{{\ensuremath{I_z}}}1
{Sx\$sym}{{\ensuremath{S_x}}}1
{Sy\$sym}{{\ensuremath{S_y}}}1
{Sz\$sym}{{\ensuremath{S_z}}}1
{p\$sym}{{\emph{p}}}1
{a\$new\$sym}{{\ensuremath{a_{\mathrm{new}}}}}1
{p\$new\$sym}{{\ensuremath{p_{\mathrm{new}}}}}1
{->}{{\ensuremath{\: \rightarrow \:}}}1
{//.}{{\ensuremath{\: //. \:}}}1  % define this before the {/.} replacement or //. will come out funny
{/.}{{\ensuremath{\: /. \:}}}1
{/@}{{\ensuremath{\: /@ \:}}}1
{Print["}{{Print[``}}8
{"Op}{{``Op}}3
{["}{{[``}}2
{,"}{{,``}}3
{A\$sym}{{\emph{A}}}1
{B\$sym}{{\emph{B}}}1
{C\$sym}{{\emph{C}}}1
{D\$sym}{{\emph{D}}}1
{sigma\$1\$sym}{{\ensuremath{\sigma_1}}}1
{aR\$sym}{{\ensuremath{a^{\dagger}}}}1
{aL\$sym}{{\ensuremath{a}}}1
{bR\$sym}{{\ensuremath{b^{\dagger}}}}1
{bL\$sym}{{\ensuremath{b}}}1
{Nop\$a}{{\ensuremath{N_a}}}1
{Nop\$b}{{\ensuremath{N_b}}}1
{\\[Alpha]}{{\ensuremath{\alpha}}}1
{\\[Beta]}{{\ensuremath{\beta}}}1
{alpha\$sym}{{\ensuremath{\alpha}}}1
{beta\$sym}{{\ensuremath{\beta}}}1
}

\section{Introduction}
% ====================
%
% \cite{Slichter1990}
%
%   Differential equation approach.
%
% ====================
%
% \cite{Shriver1991oct} = NMR product operator calculations ... 
% \cite{Guntert1993jan} = POMA = Product operator ...
% \cite{Guntert2006aug} = Symbolic NMR Product Operator Calculations 
%
%   Repeated application of a small number of rules for weakly-coupled I = 1/2 spins.
%
% \cite{Rodriguez2001oct} = Density matrix calculations in Mathematica
%
%   Two spin 1/2 particles, 4 x 4 matrix, added gradient
%
% \cite{Anand2007dec} = Simulation of steady-state {NMR} of coupled systems using Liouville
%	space and computer algebra methods
%
%   Multiple spins, steady-state solutions of LvN equation with relaxation.
%
% \cite{Jerschow2005sep} = MathNMR: Spin and spatial tensor manipulations in Mathematica
%
% 	Significant extension; move beyond weakly coupled spins and introduce spatial rotations
%   as well.  Must specify total spin angular momentum for each spin.  "Semi-symbolic".
%
% \cite{Filip2010nov} = SD-CAS: Spin Dynamics by Computer Algebra Systems
%
%	Algorithms developed for algebraically calculating products and commutators of expressions 
%   involving products of many spin operators.  
%
% \cite{Loke2011oct} = An efficient quantum circuit analyser on qubits and qudits
%
% 	"The CUGates notebook simulates arbitrarily complex quantum circuits comprised of
%	single/multiple qubit and qudit quantum gates.  It utilizes an irreducible form of
%   matrix decomposition for a general controlled gate with multiple conditionals and
%	is highly efficient in simulating complex quantum circuits." 
%
% ====================
% 
% \cite{Untidt2002jan} = Closed solution to the Baker-Campbell-Hausdorff problem: Exact 
%	effective Hamiltonian theory for analysis of nuclear-magnetic-resonance experiments
%
%	For spins, the matrices appearing in the propagator are finite and the propagator can be
%   expanded using the Cayley-Hamilton theorem.  Products of propagators can be combined using
%   the Baker-Campbell-Hausdorff, which for small numbers of spins also contains only a finite 
%   number of terms.   Goal: an effective Hamiltonian.
%
% \cite{Weyrauch2009sep} = Computing the Baker?Campbell?Hausdorff series and the Zassenhaus product
%
%	Evaluate different algorithms for writing the propagator.
%
% ====================
%
% \cite{Beskrovnyi1998jun} = Applying Mathematica to the analytical solution of the 
% 	nonlinear Heisenberg operator equations
%
% \cite{Nguyen1998dec} = Symbolic calculations of unitary transformations in quantum dynamics
%
%	Unitary transformations are represented by a truncated, finite series of nested commutators 
%	involving the generator.  Nguyen et al. described a procedure for resumming the subseries
%   to give a closed-form expression; we have tried this approach and find it to fail when the
%   arguments of the sines, cosines, and exponentials involve non-simple expressions such as
%   fractions.  Beskrovnyi1 was interested in a few coupled multiple oscillators, field modes. 
%   Same approach, but with multiple modes. 
%
% ==============
%
% \cite{Weatherford2004oct} = Symbolic calculation in chemistry: Selected examples
%
%	A wide but shallow review
%
% ==============
%
% Large packages:
%
% \cite{Helton2015feb} = NCAlgebra; impressive 
%
% \cite{Levitt2015mar} = SpinDynamica; wow ...
%
% ==============

\cite{Helton2015feb} \cite{Levitt2015mar}

\section{Operators and scalars}
% =============================

Our first task is to define functions that enable \emph{Mathematica} to distinguish between (non-commutative) operators and (commutative) scalars.  This is done in the \verb+OpCreate.m+ package, whose listing appears below.  Unit tests for the functions in this package are organized into a separate package, \verb+OpCreate-tests.m+, whose listing follows. 

\lstinputlisting[linerange=START-END,includerangemarker=false]{unidyn/OpCreate.m}
\lstinputlisting[linerange=START-END,includerangemarker=false]{unidyn/OpCreate-tests.m}

\section{Non-commutative multiplication}
% ======================================

\lstinputlisting[linerange=START-END,includerangemarker=false]{unidyn/Mult.m}
\lstinputlisting[linerange=START-END,includerangemarker=false]{unidyn/Mult-tests.m}

\section{Commutator}
% ==================

\lstinputlisting[linerange=START-END,includerangemarker=false]{unidyn/Comm.m}
\lstinputlisting[linerange=START-END,includerangemarker=false]{unidyn/Comm-tests.m}

\section{Spins}
% =============

\lstinputlisting[linerange=START-END,includerangemarker=false]{unidyn/Spins.m}
\lstinputlisting[linerange=START-END,includerangemarker=false]{unidyn/Spins-tests.m}

% Cited References
% ================

\clearpage
\bibliographystyle{bst/pccp-bibtex-nomonth-title-url-conf-note}
\renewcommand*{\bibfont}{\raggedright\normalfont\small}
\bibliography{bib/UNSORTED_bib}


\end{document}