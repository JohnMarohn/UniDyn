%!TEX root = UniDyn-doc.tex

% \cite{Slichter1990}
%
%   Differential equation approach.
%
% ====================
%
% \cite{Shriver1991oct} = NMR product operator calculations ... 
% \cite{Guntert1993jan} = POMA = Product operator ...
% \cite{Guntert2006aug} = Symbolic NMR Product Operator Calculations 
%
%   Repeated application of a small number of rules for weakly-coupled I = 1/2 spins.


% \cite{Rodriguez2001oct} = Density matrix calculations in Mathematica
%
%   Two spin 1/2 particles, 4 x 4 matrix, added gradient
%
% \cite{Anand2007dec} = Simulation of steady-state {NMR} of coupled systems using Liouville
%	space and computer algebra methods
%
%   Multiple spins, steady-state solutions of LvN equation with relaxation.
%
% \cite{Jerschow2005sep} = MathNMR: Spin and spatial tensor manipulations in Mathematica
%
%   Significant extension; move beyond weakly coupled spins and introduce spatial rotations
%   as well.  Must specify total spin angular momentum for each spin.  "Semi-symbolic".
%
% \cite{Filip2010nov} = SD-CAS: Spin Dynamics by Computer Algebra Systems
%
%   Algorithms developed for algebraically calculating products and commutators of expressions 
%   involving products of many spin operators.  
%
% \cite{Loke2011oct} = An efficient quantum circuit analyser on qubits and qudits
%
%  "The CUGates notebook simulates arbitrarily complex quantum circuits comprised of
%  single/multiple qubit and qudit quantum gates.  It utilizes an irreducible form of
%  matrix decomposition for a general controlled gate with multiple conditionals and
%  is highly efficient in simulating complex quantum circuits." 
%
% ====================
% 
% \cite{Untidt2002jan} = Closed solution to the Baker-Campbell-Hausdorff problem: Exact 
% effective Hamiltonian theory for analysis of nuclear-magnetic-resonance experiments
%
%   For spins, the matrices appearing in the propagator are finite and the propagator can be
%   expanded using the Cayley-Hamilton theorem.  Products of propagators can be combined using
%   the Baker-Campbell-Hausdorff, which for small numbers of spins also contains only a finite 
%   number of terms.   Goal: an effective Hamiltonian.
%
% \cite{Weyrauch2009sep} = Computing the Baker-Campbell-Hausdorff series and the Zassenhaus product
%
%.  Evaluate different algorithms for writing the propagator.
%
% ====================
%
% \cite{Beskrovnyi1998jun} = Applying Mathematica to the analytical solution of the 
% nonlinear Heisenberg operator equations
%
% \cite{Nguyen1998dec} = Symbolic calculations of unitary transformations in quantum dynamics
%
%   Unitary transformations are represented by a truncated, finite series of nested commutators 
%   involving the generator.  Nguyen et al. described a procedure for resumming the subseries
%   to give a closed-form expression; we have tried this approach and find it to fail when the
%   arguments of the sines, cosines, and exponentials involve non-simple expressions such as
%   fractions.  Beskrovnyi1 was interested in a few coupled multiple oscillators, field modes. 
%   Same approach, but with multiple modes. 
%
% ==============
%
% \cite{Weatherford2004oct} = Symbolic calculation in chemistry: Selected examples
%
% A wide but shallow review
%
% ==============
%
% Large packages:
%
% \cite{Helton2015feb} = NCAlgebra; impressive 
%
% \cite{Levitt2015mar} = SpinDynamica; wow ...
%
% ==============

Given a time-independent Hamiltonian ${\cal H}$ and an initial (density) operator $\rho(0)$, the \VerbFcn{Evolver} algorithm described below implements the following unitary rotation:
\begin{equation}
\rho(t) 
	= e^{-i {\cal H} t} \: \rho(0) \: e^{+i {\cal H} t}
	\equiv \text{\VerbFcn{Evolver}}[{\cal H},t,\rho(0)]
	\label{Eq:Evolve}
\end{equation}
Knowing the commutation relations among the operators comprising ${\cal H}$ and $\rho(0)$, it is often possible to obtain a closed-form algebraic solution to Eq.~\ref{Eq:Evolve}.  The \VerbFcn{Evolver} algorithm obtains a closed-form solution to Eq.~\ref{Eq:Evolve} by extending an approach introduced by Slichter in Section 2.3 of his \emph{Principles of Magnetic Resonance} text \cite{Slichter1990}.
The only input to the algorithm is the commutation relations for the relevant operators.

Slichter was interested in calculating the unitary evolution of angular momentum operators arising in spin-physics problems.  Below we introduce the Slichter procedure by considering a few representative examples.  We show that his procedure works well for harmonic oscillator problems as well, including problems involving the coupling of a harmonic oscillator with a two-level system, a minimum model for describing electron transfer in chemical reactions.  It will become apparent that automating Slichter's procedure in a computer algebra program like \emph{Mathematica} is possible but challenging.  We introduce a generalization of the Slichter algorithm that is well suited for automation by a computer algebra program.  We have implemented this algorithm as a \emph{Mathematica} function, \VerbFcn{Evolver}.  This function evaluates Eq.~\ref{Eq:Evolve} given an ${\cal H}$, a $\rho(0)$, and the commutation relations between the operators comprising ${\cal H}$ and $\rho(0)$.