%!TEX root = UniDyn-doc.tex

% \cite{Slichter1990}
%
%   Differential equation approach.
%
% ====================
%
% \cite{Shriver1991oct} = NMR product operator calculations ... 
% \cite{Guntert1993jan} = POMA = Product operator ...
% \cite{Guntert2006aug} = Symbolic NMR Product Operator Calculations 
%
%   Repeated application of a small number of rules for weakly-coupled I = 1/2 spins.
%
% \cite{Rodriguez2001oct} = Density matrix calculations in Mathematica
%
%   Two spin 1/2 particles, 4 x 4 matrix, added gradient
%
% \cite{Anand2007dec} = Simulation of steady-state {NMR} of coupled systems using Liouville
%	space and computer algebra methods
%
%   Multiple spins, steady-state solutions of LvN equation with relaxation.
%
% \cite{Jerschow2005sep} = MathNMR: Spin and spatial tensor manipulations in Mathematica
%
% 	Significant extension; move beyond weakly coupled spins and introduce spatial rotations
%   as well.  Must specify total spin angular momentum for each spin.  "Semi-symbolic".
%
% \cite{Filip2010nov} = SD-CAS: Spin Dynamics by Computer Algebra Systems
%
%	Algorithms developed for algebraically calculating products and commutators of expressions 
%   involving products of many spin operators.  
%
% \cite{Loke2011oct} = An efficient quantum circuit analyser on qubits and qudits
%
% 	"The CUGates notebook simulates arbitrarily complex quantum circuits comprised of
%	single/multiple qubit and qudit quantum gates.  It utilizes an irreducible form of
%   matrix decomposition for a general controlled gate with multiple conditionals and
%	is highly efficient in simulating complex quantum circuits." 
%
% ====================
% 
% \cite{Untidt2002jan} = Closed solution to the Baker-Campbell-Hausdorff problem: Exact 
%	effective Hamiltonian theory for analysis of nuclear-magnetic-resonance experiments
%
%	For spins, the matrices appearing in the propagator are finite and the propagator can be
%   expanded using the Cayley-Hamilton theorem.  Products of propagators can be combined using
%   the Baker-Campbell-Hausdorff, which for small numbers of spins also contains only a finite 
%   number of terms.   Goal: an effective Hamiltonian.
%
% \cite{Weyrauch2009sep} = Computing the Baker?Campbell?Hausdorff series and the Zassenhaus product
%
%	Evaluate different algorithms for writing the propagator.
%
% ====================
%
% \cite{Beskrovnyi1998jun} = Applying Mathematica to the analytical solution of the 
% 	nonlinear Heisenberg operator equations
%
% \cite{Nguyen1998dec} = Symbolic calculations of unitary transformations in quantum dynamics
%
%	Unitary transformations are represented by a truncated, finite series of nested commutators 
%	involving the generator.  Nguyen et al. described a procedure for resumming the subseries
%   to give a closed-form expression; we have tried this approach and find it to fail when the
%   arguments of the sines, cosines, and exponentials involve non-simple expressions such as
%   fractions.  Beskrovnyi1 was interested in a few coupled multiple oscillators, field modes. 
%   Same approach, but with multiple modes. 
%
% ==============
%
% \cite{Weatherford2004oct} = Symbolic calculation in chemistry: Selected examples
%
%	A wide but shallow review
%
% ==============
%
% Large packages:
%
% \cite{Helton2015feb} = NCAlgebra; impressive 
%
% \cite{Levitt2015mar} = SpinDynamica; wow ...
%
% ==============

\cite{Helton2015feb} \cite{Levitt2015mar}

Given a Hamiltonian operator ${\cal H}$ and an initial (density) operator $\rho(0)$, the \VerbFcn{Evolve} function tries to implement the following unitary rotation:
\begin{equation}
\rho(t) 
	= e^{-i {\cal H} t} \: \rho(0) \: e^{+i {\cal H} t}
	\equiv \text{\VerbFcn{Evolver}}[{\cal H},t,\rho_0]
	\label{Eq:Evolve}
\end{equation}
assuming that the Hamiltonian operator is time-independent.  To simplify Eq.~\ref{Eq:Evolve} 
any further we must specify ${\cal H}$ and $\rho(0)$.  In general ${\cal H}$ and $\rho(0)$ will involve non-commuting operators.  With the commutation relations specified, however, it is often possible to obtain a closed-form algebraic solution to Eq.~\ref{Eq:Evolve}.  Here we follow the general procedure for obtaining a closed-form solution to Eq.~\ref{Eq:Evolve} outlined by Slichter in Section 2.3 of his \emph{Principles of Magnetic Resonance} text \cite{Slichter1990}.

Slichter was interested in calculating the unitary evolution of angular momentum operators arising in spin-physics problems.  Below we introduce the Slichter procedure by considering a few representative examples.  We show that his procedure works remarkably well for calculating the unitary evolution in other cases involving, for example, the time evolution of position and momentum operators under the harmonic-oscillator Hamiltonian.  It will become apparent that automating Slichter's procedure in a computer algebra program like \emph{Mathematica} is possible but challenging.  We introduce a generalization of the Slichter algorithm that is well suited for automation by a computer algebra program.  We have implemented this algorithm as a \emph{Mathematica} function, \VerbFcn{Evolver}.  This function evaluates Eq.~\ref{Eq:Evolve} given an ${\cal H}$, a $\rho(0)$, and the commutation relations between the operators contained in ${\cal H}$ and $\rho(0)$.

To understand Slichter's procedure, let us consider first the case where ${\cal H} = \omega I_z$ and $\rho_0 = I_{+} = I_x + i \, I_y$,
\begin{equation}
\rho(t) 
	= e^{-i \, \omega t \, I_z} \: I_{+} \: e^{+i \, \omega t \, I_z}.
\end{equation}
Taking the time derivative we obtain
\begin{subequations}  
\begin{align}
\dot{\rho}(t)
	& = e^{-i \, \omega t \, I_z} \: 
		(-i \, \omega [I_z,I_{+}] ) 
		\: e^{+i \, \omega t \, I_z} 
		\label{Eq:I+rot-(a)} \\
	& = -i \, \omega \left( 
			e^{-i \, \omega t \, I_z} \: I_{+} \: e^{+i \, \omega t \, I_z} 
		\right)
		\label{Eq:I+rot-(b)}
\end{align}
\end{subequations} 
where we have used $[I_z,I_{+}] = I_{+}$ to reduce the commutator in Eq.~\ref{Eq:I+rot-(a)}.  We recognize the term in parenthesis in Eq.~\ref{Eq:I+rot-(b)} as the original time dependent density operator, $\rho(t)$.  Equation~\ref{Eq:I+rot-(b)} thus becomes
\begin{equation}
\dot{\rho}(t) 
	= -i \, \omega \, \rho(t)
	\label{Eq:Eq:I+rot-ODE}
\end{equation}
In this differential equation, $\omega$ is a \emph{number}, while $\rho(t)$ is an \emph{operator}.  The solution to this differential equation is
\begin{equation}
\rho(t) 
	= \rho(0) \, e^{-i \, \omega t}
 	= I_{+} \, e^{-i \, \omega t},
\end{equation}
which we can easily verify by back substitution.  As a result of the unitary evolution under the Hamiltonian, the $I_{+}$ operator has picked up a phase factor. An analogous calculation arises in a harmonic oscillator problem where the Hamiltonian is ${\cal H} = \omega (a^{\dagger} a + 1/2)$ and the initial operator is $\rho(0) = a^{\dagger}$, 
\begin{equation}
\rho(t) 
	= e^{-i \omega t (a^{\dagger} a + 1/2)} 
		\, a^{\dagger} 
		\, e^{+i \omega t (a^{\dagger} a + 1/2)}
\end{equation}
Using the same procedure and the commutation relation $[a^{\dagger} a,a^{\dagger}]=1$, this equation reduces to 
\begin{equation}
	\rho(t)  = a^{\dagger} e^{-i \: \omega t}.
\end{equation}
These two cases have in common that the commutator of the Hamiltonian with the operator of interest is simply proportional to the operator. As a result of this underlying commutation relation, the problem of calculating Eq.~\ref{Eq:Evolve} in both cases has been reduced to the problem of solving a first-order differential equation, Eq.~\ref{Eq:Eq:I+rot-ODE}.  

The second case to consider is ${\cal H} = \omega I_z$ and $\rho_0 = I_x$,
\begin{equation}
\rho(t) 
	= e^{-i \, \omega t \, I_z} \: I_x \: e^{+i \, \omega t \, I_z}.
\end{equation}
Taking the time derivative we obtain
\begin{subequations}  
\begin{align}
\dot{\rho}(t)
	& = e^{-i \, \omega t \, I_z} \: 
		(-i \, \omega [I_z,I_x] ) 
		\: e^{+i \, \omega t \, I_z} 
		\label{Eq:Ixrot-(a)} \\
	& = \omega \left( 
			e^{-i \, \omega t \, I_z} \: I_y \: e^{+i \, \omega t \, I_z} 
		\right)
		\label{Eq:Ixrot-(b)}
\end{align}
\end{subequations} 
where we have used $[I_z,I_x] = i \, I_y$ to reduce the commutator in Eq.~\ref{Eq:Ixrot-(a)}.  In contrast to the previous case, $\dot{\rho}(t)$ is not proportional to $\dot{\rho}$.  Taking another time derivative we obtain
\begin{subequations}  
\begin{align}
\ddot{\rho}(t)
	& = \omega \, e^{-i \, \omega t \, I_z} \: 
		(-i \, \omega [I_z,I_y] ) 
		\: e^{+i \, \omega t \, I_z} 
		\label{Eq:Ixrot-(c)} \\
	& = \omega^2 \left( 
			e^{-i \, \omega t \, I_z} \: I_x \: e^{+i \, \omega t \, I_z} 
		\right)
		\label{Eq:Ixrot-(d)}
\end{align}
\end{subequations} 
where we have used $[I_z,I_y] = -i \, I_x$ to reduce the commutator in Eq.~\ref{Eq:Ixrot-(a)}. The term in parenthesis in Eq.~\ref{Eq:Ixrot-(d)} is proportional to $\rho(t)$ and consequently
\begin{equation}
\ddot{\rho}(t) = \omega^2 \, \rho(t)
\end{equation}
The solution to this differential equation is
\begin{equation}
\rho(t) = \rho(0) \, \sin{(\omega t)} + \frac{\dot{\rho}(0)}{\omega} \, \cos{(\omega t)}
\end{equation}
We are given that $\rho(0) = I_x$ and we see from Eq.~\ref{Eq:Ixrot-(b)} that $\dot{\rho}(0) = \omega \, I_y$.  Plugging in these initial conditions we obtain
\begin{equation}
\rho(t) = I_x \, \sin{(\omega t)} + I_y \, \cos{(\omega t)}
\end{equation}

