%!TEX root = UniDyn-doc.tex


Given a time-independent Hamiltonian ${\cal H}$ and an initial density operator $\rho(0)$ evolving in the Schr\"{o}dinger represenation, the \VerbFcn{Evolver} algorithm described below computes the density operator at time $t$ by implementing the following unitary rotation
\begin{equation}
\rho(t) 
	= e^{-i {\cal H} t} \: \rho(0) \: e^{+i {\cal H} t}
	\equiv \text{\VerbFcn{Evolver}}[{\cal H},t,\rho(0)]
	\label{Eq:Evolve}
\end{equation}
The same algorithm can be used to compute the time dependence of an operator evolving in the Heinsenberg representation by replacing $t$ with $-t$ and $\rho(0)$ with the operator of interest. 

Consider two representative examples.
In the Schr\"{o}dinger representation, the time evolution of transverse spin magnetization in a longitudinal magnetic field is described by the rotation
\begin{equation}
e^{-i \, \omega t \, I_z} \: 
  I_x \: 
  e^{+i \, \omega t \, I_z} 
   = I_x \, \sin{(\omega t)} + I_y \, \cos{(\omega t)}
   \label{eq:example-spin}
\end{equation}
where $\omega$ is the Larmor frequency and $I_x$, $I_y$, and $I_z$ are the spin angular-momentum operators.  In the Heisenberg representation, the harmonic oscillator's creation operator $a^{\dagger}$ evolves in time under the harmonic oscillator's Hamiltonian as follows:
 \begin{equation}
 e^{+ i \omega t \frac{1}{2} (a^{\dagger} \, a + a \, a^{\dagger})} 
   a^{\dagger}
   e^{- i \omega t \frac{1}{2} (a^{\dagger} \, a + a \, a^{\dagger})} 
   =  a^{\dagger} e^{+i \omega t}
   \label{eq:example-ho}
\end{equation}
where $a$ is the annihilation operator and $\omega$ is the harmonic-oscillator's resonance frequency.
Knowing the commutation relations among the operators comprising ${\cal H}$ and $\rho(0)$, it is often possible to obtain a closed-form algebraic solution to Eq.~\ref{Eq:Evolve}.  
The \VerbFcn{Evolver} algorithm obtains closed-form solutions to Eq.~\ref{Eq:Evolve} by extending an approach introduced by Slichter in Section 2.3 of his \emph{Principles of Magnetic Resonance} text \cite{Slichter1990}.
In this approach, one infers the differential equation that $\rho(t)$ satisfies.
The only input to the algorithm is the commutation relations for the relevant operators.

\section{Prior work}

Numerous software programs exist for computing eq.~\ref{Eq:Evolve} dynamics numerically.
In order to gain physical understanding, researchers want to compute the dynamics symbolically, using computer algebra programs.
This effort has been motivated by applications in magnetic resonance \cite{Shriver1991oct,Kanters1993jan,Guntert1993jan,Isbister1995dec,Rodriguez2001jan,Guntert2006aug,Anand2007dec,Kuprov2007feb,Filip2010nov}, 
quantum computing \cite{Loke2011oct,Chen2013mar,Loke2013dec},
electronic structure \cite{Zitko2011oct},
vibrational dynamics \cite{Aleixo2011aug}, and 
quantum optics \cite{Beskrovnyi1998jun,Nguyen1998dec}.
Unitary rotations like eq.~\ref{Eq:Evolve} occur in exact effective Hamiltonian theory \cite{Untidt2002jan,Siminovitch2004jan}, 
the Baker-Campbell-Hausdorff formula \cite{Weyrauch2009sep,Aleixo2011aug},
the Zassenhaus product formula \cite{Weyrauch2009sep,Aleixo2011aug, Casas2012nov}, 
the Hadamard lemma \cite{Aleixo2011aug}, and 
simplifications of exponential operators involving raising and lower operators that are possible if the underlying potential has certain symmetries \cite{Aleixo2011aug}.

Most eq.~\ref{Eq:Evolve} work to date, motivated by applications in magnetic resonance 
% \cite{Isbister1995dec,Rodriguez2001jan,Jerschow2005sep,Kuprov2007feb,Anand2007dec,Bengs2017sep} 
and quantum computing% 
% \cite{Loke2011oct,Chen2013mar,Loke2013dec}%
, has been semi-symbolic.
Spins and few-level electron systems have a finite Hilbert space and can be represented by finite-dimensional matrices.
In the semi-symbolic approaches, spin-angular momentum operators are expressed as matrices \cite{Isbister1995dec,Rodriguez2001jan,Loke2011oct,Chen2013mar,Loke2013dec,Bengs2017sep}
or irreducible spherical tensors \cite{Jerschow2005sep,Kuprov2007feb,Anand2007dec,Bengs2017sep},
while Hamiltonian and irradiation parameters remain symbolic. 
This semi-symbolic approach has led to powerful \emph{Mathematica} packages like \emph{MathNMR} \cite{Jerschow2005sep} and \emph{SpinDynamica} \cite{Bengs2017sep}.

Vibrations and photons have an infinite (or near-infinite) Hilbert space and consequently the associated creation and annihilation operators do not have a matrix representation.
Evolving creation and annihilation operators requires a purely symbolic approach.

There are fewer examples of purely symbolic approaches to computing eq.~\ref{Eq:Evolve} dynamics.
We know of only one algorithm that attempted to calculate commutators of spin- and harmonic-oscillator operators using a purely symbolic approach \cite{Zitko2011oct}, and this algorithm stopped short of considering unitary evolution.
In magnetic resonance, spin angular momenta have been described using non-commuting product operators, with evolution under pulsed irradiation and free evolution implemented either approximately, using a Baker-Campbell-Hausdorff (BCH) expansion \cite{Filip2010nov}, or exactly using operator substitution rules \cite{Shriver1991oct,Kanters1993jan,Guntert1993jan,Guntert2006aug}.
Computer algebra programs have been used to carry out the multiplication and commutation of harmonic-oscillator (\emph{i.e.}\ boson) operators symbolically \cite{Beskrovnyi1998jun,Nguyen1998dec,Zitko2011oct}, but eq.~\ref{Eq:Evolve} dynamics was only computed approximately using a BCH expansion, with the unitary transformation represented by a truncated, finite series of nested commutators.
Nguyen and workers described a promising procedure for resumming the resulting finite series to give a closed-form expression \cite{Nguyen1998dec}.
We find that this approach fails when the arguments of the associated sines, cosines, and exponentials involve non-trivial expressions such as fractions.

Slichter was interested in calculating the unitary evolution of angular momentum operators arising in spin-physics problems \cite{Slichter1990}, like the rotation in eq.~\ref{eq:example-spin}.
In the following section we introduce the Slichter procedure by considering example cases.
We show that his procedure works well for harmonic oscillator problems too, including problems involving the coupling of a harmonic oscillator with a two-level system, a minimum model for describing electron transfer in chemical reactions and radiation-matter interactions in the strong-coupling limit.

It will become apparent that automating Slichter's procedure in a computer algebra program like \emph{Mathematica} is possible but challenging.
There are two challenges.
The first is implementing non-commutative algebra in \emph{Mathematica} in a useful way. 
The second is defining an operator inverse.
We address these problems below.
We then introduce two generalizations of the Slichter algorithm that are well suited for automation by a computer algebra program.
We have implemented these algorithm as a \emph{Mathematica} function, \VerbFcn{Evolver}.
This function evaluates Eq.~\ref{Eq:Evolve} given an ${\cal H}$, an initial operator $\rho(0)$, and the commutation relations between the operators comprising ${\cal H}$ and $\rho(0)$.

% An exponentiated spin operator $A$ of the form $e^{i A}$ can be written, using the Cayley-Hamilton theorem, as a polynomial function of $A$ \cite{DeZela2014jun,Curtright2014aug}.

% Spin $I = 1/2$ operators, Pauli matrices.

% Spin $I = 1$ operators, the eight Gell-Mann matrices \cite{Gell-Mann1962feb,Bertlmann2008may,Curtright2015dec}